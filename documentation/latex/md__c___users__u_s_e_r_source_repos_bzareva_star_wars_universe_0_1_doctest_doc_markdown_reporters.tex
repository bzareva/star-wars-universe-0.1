Doctest has a modular reporter/listener system with which users can write their own reporters and register them. The reporter interface can also be used for \char`\"{}listening\char`\"{} to events.

You can list all registered reporters/listeners with {\ttfamily -\/-\/list-\/reporters}. There are a few implemented reporters in the framework\+:
\begin{DoxyItemize}
\item {\ttfamily console} -\/ streaming -\/ writes normal lines of text with coloring if a capable terminal is detected
\item {\ttfamily xml} -\/ streaming -\/ writes in xml format tailored to doctest
\item {\ttfamily junit} -\/ buffering -\/ writes in JUnit-\/compatible xml -\/ for more information look \href{https://github.com/onqtam/doctest/issues/318}{\texttt{ here}} and \href{https://github.com/onqtam/doctest/issues/376}{\texttt{ here}}.
\end{DoxyItemize}

Streaming means that results are delivered progressively and not at the end of the test run.

The output is by default written to {\ttfamily stdout} but can be redirected with the use of the {\ttfamily -\/-\/out=$<$filename$>$} \mbox{\hyperlink{md__c___users__u_s_e_r_source_repos_bzareva_star_wars_universe_0_1_doctest_doc_markdown_commandline}{$\ast$$\ast$command line option$\ast$$\ast$}}.

Example how to define your own reporter\+:


\begin{DoxyCode}{0}
\DoxyCodeLine{ \{c++\}}
\DoxyCodeLine{\#include <doctest/doctest.h>}
\DoxyCodeLine{}
\DoxyCodeLine{\#include <mutex>}
\DoxyCodeLine{}
\DoxyCodeLine{using namespace doctest;}
\DoxyCodeLine{}
\DoxyCodeLine{struct MyXmlReporter : public IReporter}
\DoxyCodeLine{\{}
\DoxyCodeLine{    // caching pointers/references to objects of these types -\/ safe to do}
\DoxyCodeLine{    std::ostream\&         stdout\_stream;}
\DoxyCodeLine{    const ContextOptions\& opt;}
\DoxyCodeLine{    const TestCaseData*   tc;}
\DoxyCodeLine{    std::mutex            mutex;}
\DoxyCodeLine{}
\DoxyCodeLine{    // constructor has to accept the ContextOptions by ref as a single argument}
\DoxyCodeLine{    MyXmlReporter(const ContextOptions\& in)}
\DoxyCodeLine{            : stdout\_stream(*in.cout)}
\DoxyCodeLine{            , opt(in) \{\}}
\DoxyCodeLine{}
\DoxyCodeLine{    void report\_query(const QueryData\& /*in*/) override \{\}}
\DoxyCodeLine{}
\DoxyCodeLine{    void test\_run\_start() override \{\}}
\DoxyCodeLine{}
\DoxyCodeLine{    void test\_run\_end(const TestRunStats\& /*in*/) override \{\}}
\DoxyCodeLine{}
\DoxyCodeLine{    void test\_case\_start(const TestCaseData\& in) override \{ tc = \&in; \}}
\DoxyCodeLine{}
\DoxyCodeLine{    // called when a test case is reentered because of unfinished subcases}
\DoxyCodeLine{    void test\_case\_reenter(const TestCaseData\& /*in*/) override \{\}}
\DoxyCodeLine{}
\DoxyCodeLine{    void test\_case\_end(const CurrentTestCaseStats\& /*in*/) override \{\}}
\DoxyCodeLine{}
\DoxyCodeLine{    void test\_case\_exception(const TestCaseException\& /*in*/) override \{\}}
\DoxyCodeLine{}
\DoxyCodeLine{    void subcase\_start(const SubcaseSignature\& /*in*/) override \{}
\DoxyCodeLine{        std::lock\_guard<std::mutex> lock(mutex);}
\DoxyCodeLine{    \}}
\DoxyCodeLine{}
\DoxyCodeLine{    void subcase\_end() override \{}
\DoxyCodeLine{        std::lock\_guard<std::mutex> lock(mutex);}
\DoxyCodeLine{    \}}
\DoxyCodeLine{}
\DoxyCodeLine{    void log\_assert(const AssertData\& in) override \{}
\DoxyCodeLine{        // don't include successful asserts by default -\/ this is done here}
\DoxyCodeLine{        // instead of in the framework itself because doctest doesn't know}
\DoxyCodeLine{        // if/when a reporter/listener cares about successful results}
\DoxyCodeLine{        if(!in.m\_failed \&\& !opt.success)}
\DoxyCodeLine{            return;}
\DoxyCodeLine{}
\DoxyCodeLine{        // make sure there are no races -\/ this is done here instead of in the}
\DoxyCodeLine{        // framework itself because doctest doesn't know if reporters/listeners}
\DoxyCodeLine{        // care about successful asserts and thus doesn't lock a mutex unnecessarily}
\DoxyCodeLine{        std::lock\_guard<std::mutex> lock(mutex);}
\DoxyCodeLine{}
\DoxyCodeLine{        // ...}
\DoxyCodeLine{    \}}
\DoxyCodeLine{}
\DoxyCodeLine{    void log\_message(const MessageData\& /*in*/) override \{}
\DoxyCodeLine{        // messages too can be used in a multi-\/threaded context -\/ like asserts}
\DoxyCodeLine{        std::lock\_guard<std::mutex> lock(mutex);}
\DoxyCodeLine{}
\DoxyCodeLine{        // ...}
\DoxyCodeLine{    \}}
\DoxyCodeLine{}
\DoxyCodeLine{    void test\_case\_skipped(const TestCaseData\& /*in*/) override \{\}}
\DoxyCodeLine{\};}
\DoxyCodeLine{}
\DoxyCodeLine{// "{}1"{} is the priority -\/ used for ordering when multiple reporters are used}
\DoxyCodeLine{REGISTER\_REPORTER("{}my\_xml"{}, 1, MyXmlReporter);}
\DoxyCodeLine{}
\DoxyCodeLine{// registering the same class as a reporter and as a listener is nonsense but it's possible}
\DoxyCodeLine{REGISTER\_LISTENER("{}my\_listener"{}, 1, MyXmlReporter);}

\end{DoxyCode}


Custom {\ttfamily IReporter} implementations must be registered with one of\+:


\begin{DoxyItemize}
\item {\ttfamily REGISTER\+\_\+\+REPORTER}, for when the new reporter is an option that users may choose at run-\/time.
\item {\ttfamily REGISTER\+\_\+\+LISTENER}, for when the reporter is actually a listener and must always be executed, regardless of which reporters have been chosen at run-\/time.
\end{DoxyItemize}

Multiple reporters can be used at the same time -\/ just specify them through the {\ttfamily -\/-\/reporters=...} \mbox{\hyperlink{md__c___users__u_s_e_r_source_repos_bzareva_star_wars_universe_0_1_doctest_doc_markdown_commandline}{$\ast$$\ast$command line filtering option$\ast$$\ast$}} using commas to separate them like this\+: {\ttfamily -\/-\/reporters=my\+Reporter,xml} and their order of execution will be based on their priority -\/ that is the number \char`\"{}1\char`\"{} in the case of the example reporter above (lower means earlier -\/ the default console/xml reporters from the framework have 0 as their priority and negative numbers are accepted as well).

All registered listeners ({\ttfamily REGISTER\+\_\+\+LISTENER}) will be executed before any reporter -\/ they do not need to be specified and cannot be filtered through the command line.

When implementing a reporter users are advised to follow the comments from the example above and look at the few implemented reporters in the framework itself. Also check out the \href{../../examples/all_features/reporters_and_listeners.cpp}{\texttt{ {\bfseries{example}}}}.

\DoxyHorRuler{0}


\href{readme.md\#reference}{\texttt{ Home}}



